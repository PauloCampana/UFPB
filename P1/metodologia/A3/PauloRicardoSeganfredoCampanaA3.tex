\documentclass[12pt]{article}   % artigo fonte 12
\usepackage[
a4paper,		% papel A4
left=1.5cm,		% margem esquerda
right=1.5cm,		% margem direita
top=1.5cm,		% margem superior
bottom=1.5cm		% margem inferior
]{geometry}
% Caracteres / Acentos / Português do Brasil
\usepackage{inputenc}[utf8]
\usepackage{fontenc}[T1]
\usepackage{babel}[brazil]
% Pacotes Matemática
\usepackage{array,latexsym}
\usepackage{amsmath,amsfonts,amssymb,amsthm,mathabx,amstext}
\usepackage{dsfont}	% Conjuntos: $\mathds{N, Z, Q, R, C}$
\usepackage{graphicx}

\begin{document}

	\begin{table}[h!]
		\centering
		\caption{Distribuição da população ocupada e da população subocupada por insuficiência de horas (valores absolutos por 1 000 pessoas), segundo grupos de atividades - 2019}
		\label{tab_ocup}
		\begin{tabular}{l|rr}
			\hline
			\multicolumn{1}{c|}{\textbf{Grupos de atividade (1)}}&\textbf{Ocupada}&\textbf{Subocupada (2)}\\
			\hline
			&&\\
			\textbf{Total}&94 642&7 187\\
			&&\\
			Agropecuária&8 652&941\\
			Indústria&12 231&462\\
			Construção&6 823&692\\
			Comércio e reparação&17 914&949\\
			Adm. pública, educação, saúde e serv. sociais&16 459&936\\
			Transporte, armazenagem e correio&4 841&256\\
			Alojamento e alimentação&5 631&554\\
			Informação, financeira e outras ativ. profissionais&10 716&540\\
			Serviços domésticos&6 271&1 129\\
			Outros serviços&5 068&724\\
			&&\\
			\hline
			\multicolumn{3}{l}{\textbf{Fonte:} IBGE, PNAD, 2019.}\\
			\multicolumn{3}{l}{\textbf{Notas:} (1) Não são apresentados resultados para atividades mal definidas.}\\
			\multicolumn{3}{l}{(2) dados calculados pelas horas trabalhadas até setembro/2015}\\
		\end{tabular}
	\end{table}

	\begin{table}[h!]
		\centering
		\caption{População com 14 anos ou mais de idade ocupadas na semana de referência (valores absolutos por 1 000 pessoas), segundo sexo e posição na ocupação e trabalho formal ou informal - 2019}
		\label{tab_ocupteen}
		\begin{tabular}{l|rr}
			\hline
			\multicolumn{1}{c|}{\textbf{Posição na ocupação}}&\multicolumn{2}{c}{\textbf{Sexo}}\\
			&\textbf{Homem}&\textbf{Mulher}\\
			\hline
			&&\\
			\textbf{Total}&53 306&41 336\\
			&&\\
			\textbf{Formal}&31 210&24 086\\
			Empregado com carteira&20 747&14 460\\
			Trabalhador doméstico com carteira&198&1 528\\
			Militar ou funcionário público&3 425& 4 446\\
			Conta própria contribuinte&4 736&2 620\\
			Empregador contribuinte&2 104&1 031\\
			&&\\
			\textbf{Informal}&22 096&17 250\\
			Empregado sem carteira&9 027&5 455\\
			Trabalhador doméstico sem carteira&290&4 196\\
			Conta própria não contribuinte&11 067&5 993\\
			Empregador não contribuinte&930&305\\
			Trabalhador familiar auxiliar&784&1 300\\
			&&\\
			\hline
			\multicolumn{3}{l}{\textbf{Fonte:} IBGE, PNAD, 2019.}\\
			\multicolumn{3}{l}{\textbf{Nota:} A classificação dos trabalhos formais/informais}\\
			\multicolumn{3}{l}{baseia-se na definição da Org. Int. do Trabalho - OTI.}
		\end{tabular}
	\end{table}










\end{document}