\documentclass[12pt]{article}
\usepackage[utf8]{inputenc}
\usepackage[T1]{fontenc}	
\usepackage[brazil]{babel}
\usepackage{array,latexsym}
\usepackage{amsmath,amsfonts,amssymb,
amsthm,mathabx,amstext}
\usepackage{dsfont}
\usepackage{graphicx}	
\usepackage[alf]{abntex2cite}

\title{Análise descritiva dos dados de homicídios ocorridos em João Pessoa - PB,\\
no período de 1996 à 2004}
\author{Mônica Maria Ferreira Teles, Izabel Cristina Alcantara de Souza,\\
 Ronei Marcos de Moraes}
\date{Departamento de Estatística, Universidade Federal da Paraíba
\url{monifpb@ig.com.br}, \url{souza_izabel@yahoo.com}, \url{ronei@de.ufpb.br}}
\begin{document}
\maketitle
\section*{Resumo}
%%%%%%%%%%%%%%%%%%%%%%%%%%%%%%%%%%%
%%% Atividade B3
% Crie as referências e citações usando abntex2cite
%
% Este resumo tem 6 referências, numeradas de 1 à 6,
% cada uma correspondente a um artigo em anexo.
%
% A atividade B3 consistem em trocar cada citação/referência
% pela entrada do LaTeX, isto é, aqui colocar a citação
% no arquivo referencias.bib colocar a referência
%
% Vamos a um Exemplo, com a citação [1]
% A frase começa com:
% Briceño-Leon [1] destaca que o aumento do percentual....
%
% Deve ser substituída por:
% \citeonline{BricenoLeon} destaca que o aumento do percentual....
%
% No arquivo referencias.bib já foi adicionado, como modelo, a referência:
%@article{
%BricenoLeon,
%author={R. {Brice\~no-Le\'on}},
%journal={Sociologias},
%title={La nueva violencia urbana de América Latina},
%subtitle={ },
%volume={ },
%number={8},
%pages={34--51},
%year={2002}
%}
%
% Para as outras, copie e substitua as informações.
% Só precisa substituir estes elementos,
% se não tem algum, deixe em branco.
% 
% Como o sobrenome é composto: "Briceño-León", 
% coloquei entre chaves: R. {Briceño-León}
%
% author={R. {Briceño-León}},
%
% Para que o abntex2cite entenda que não é só León, 
% mas sim Briceño-León
%
% Se para você os acentos ficaram errados,
% então use os acentos como explicado na videoaula
%
% author={R. {Brice\~no-Le\'on}},
%
%%%%%%%%%%%%%%%%%%%%%%%%%%%%%%%%%%%%%%%%%%%%%%
Do total dos homicídios ocorridos na América Latina aproximadamente 80\% são ocasionados
através de armas de fogo. \citeonline{BricenoLeon} destaca que o aumento do percentual desta
modalidade de homicídios não está vinculado a um aumento dos
conflitos geradores das agressões, mas a um aumento na posse de armas de fogo, cuja
eficiência para matar é maior do que as armas brancas. \citeonline{Minayo} cita que no Brasil na
 década de 80 os homicídios como causa
específica apresentavam uma maior tendência de crescimento. O referido trabalho já
associava este fato ao incremento do uso de armas de fogo, que foram os instrumentos usados
em 47,3\% dos casos de homicídios notificados. Para os dados do ano de 2000, \citeonline{Gawryszewski} afirmam que os homicídios
lideram os tipos de mortalidade com 45.343 vítimas no Brasil (35,1\% homens e 3,2\%
mulheres), deste total 63,5\% foram vitimadas por armas de fogo. Em Pernambuco, \citeonline{Lima} estudando o período compreendido entre
1980 e 1998, encontrou para homens de idade entre 15 e 49 anos uma taxa de mortalidade de
81,2 por 100.000 habitantes. Em termos comparativos esta taxa já ultrapassa as encontradas
para a população total da Colômbia no ano de 1994, que fora de 78,5 por 100.000 habitantes.
\par Neste contexto e diante do fato de haver poucas análises sobre o tema para o âmbito da
Paraíba, este trabalho objetiva realizar uma Análise Descritiva dos dados de homicídios
ocorridos em João Pessoa-PB, no período de 1996 a 2004. Os dados foram obtidos do SIM -
Sistema de Informações Sobre Mortalidade. Este é um sistema nacional de vigilância
epidemiológica cuja fonte de dados são as declarações de óbito (DO), de onde os dados são
compilados a partir das informações municipais, agregando-se posteriormente a nível estadual
e nacional. As variáveis utilizadas foram: a causa básica de morte, idade, sexo, escolaridade e
local de ocorrência. Foram selecionadas as causas de óbitos especificadas na CID-10 -
Classificação Estatística Internacional de Doenças e Problemas Relacionados à Saúde
(Décima Revisão). Selecionaram-se as causas listadas no capítulo XX - Causas externas de
morbidade e mortalidade, mais especificamente do grupo de causas classificado como
Agressões, incluindo homicídio, lesões infligidas por outra pessoa, empregando qualquer
meio, com a intenção de lesar (ferir) ou de matar. As causas selecionadas para este estudo
estão incluídas na lista da CID através da nomenclatura que vai do X85 a Y09 que tratam
especificamente de homicídios.
\par Os resultados apontam que em João Pessoa no período analisado houve percentuais maiores
que 70\% para homicídios efetivados através de armas de fogo (CID-X93-95), exceto para o
ano de 1996 com 42,4\%. Depois das agressões por armas de fogo, as maiores freqüências
foram observadas para agressões por meio de objetos cortantes ou perfurantes (CID-X99) que
apresentaram percentuais próximos a 15\%, com exceção do ano de 1996 onde o percentual
ficou em 40,1\%. Para os outros tipos de homicídios, além de percentuais menores que 9\%,
houve ausência de notificação (falta de ocorrência) de alguns tipos por anos seguidos.
Verifica-se portanto, que em 1996, os homicídios causados por arma de fogo e por objetos
cortantes apresentavam percentuais de 42,4\% e 40,1\% respectivamente. A partir de 1997,
observa-se uma mudança de padrão, com os percentuais para objetos cortantes e perfurantes
variando entre 12,1\% e 24,6\%, respectivamente em 2003 e 1999. Já os homicídios por armas
de fogo apresentaram percentuais mínimo e máximo de 70,5\% (em 1999) e 83,6\% (em 2000)
respectivamente.
\par Com relação aos fatores demográficos, pode-se observar que as maiores vítimas de
homicídios encontram-se na faixa etária de 16 a 34 anos, com percentuais mínimo e máximo
para o período de 66,1\% (em 1996) e 77,8\% (em 2000) respectivamente. Quanto ao gênero,
houve uma predominância para o sexo masculino, com percentuais mínimo e máximo para o
período de 86,9\% (em 1999) e 94,8\% (em 2003) respectivamente.
\par Para os dados de escolaridade não houve registro para o período de 1996 a 1998, analisando-se,
portanto, o período de 1999 a 2004. Os percentuais mínimos e máximos para as faixas de
escolaridade das vítimas foram os seguintes: 23,3\% (em 2004) e 48,6\% (em 2001) para a
faixa de 1 a 3 anos de estudos concluídos; 14,3\% (em 1999) e 30,8\% (em 2001) para a faixa
de 4 a 7 anos de estudos concluídos; zero (em 2004) e 11,4\% (em 1999) para a faixa de 8 a 11
anos de estudos concluídos e 0,9\% (em 2000) e 2,9\% (em 1999) para a faixa de 12 ou mais
anos de estudos concluídos. No decorrer do período analisado, para as vítimas registradas
com nenhuma escolaridade, observaram-se percentuais variando entre 3,3\% (em 2004) e
22,9\% (em 1999).
\par Quanto ao local de ocorrência, para o período de 1996 a 2004, verificou-se que entre 30,1\%
(em 2003) e 43,3\% (em 1996) dos homicídios ocorrem em vias públicas e cerca de 25\% (em
1998) a 39,5\% (em 2004) por cento ocorrem em hospital. Os locais de ocorrência
considerados como “outros”, que incluem casas de shows, bares, feiras livres, etc,
apresentaram percentuais que variam entre 5,2\% (em 2001) e 30,8\% (em 1998). Os
percentuais para homicídios ocorridos em domicílios ficaram em valores mínimos e máximos
de 4,9\% (em 2000) e 21,3\% (em 1999) respectivamente.
\par A despeito do caráter superficial das análises aqui empreendidas, considerando o sexo e a
faixa etária das vítimas, os resultados indicam que as principais vítimas são homens em idade
produtiva, semelhante ao que aponta \citeonline{Xiemenes} e \citeonline{Reichenheim}. Destacou-se neste estudo a escolaridade como variável importante em análises
sobre violência urbana. De acordo com os resultados obtidos para o período de 1999 a 2004
em João Pessoa, pode-se afirmar que quanto mais anos de estudos concluídos menores são os
percentuais de vítimas de óbitos por causas violentas. No mais, este estudo é o primeiro de
uma série que analisará a situação da violência urbana em João Pessoa – PB, destacando a
carência de estudos sobre este tema no Estado.\\
\vspace{12pt}
\par\noindent \textbf{Palavras-chave:} Causas Externas, Agressões, Violência, Homicídios.
%%%%%%%%%%%%%%%%%%%%%%%%%%%
%% Entrada da referência %%
\bibliography{referencias.bib}
\end{document}