\documentclass[12pt]{article}
\usepackage[utf8]{inputenc}
\usepackage[T1]{fontenc}	
\usepackage{array,latexsym}
\usepackage{amsmath,amsfonts,amssymb,amsthm,mathabx,amstext}
\usepackage{dsfont}	% Conjuntos: $\mathds{N, Z, Q, R, C}$
\usepackage{graphicx}
\usepackage{qtree}
\begin{document}
	\pagestyle{empty}
	\begin{center}
		Atividade 2 - Propriedades da probabilidade e probabilidade condicional
		Paulo Ricardo Seganfredo Campana\\
	\end{center}
	
1.\\
a) $\lbrace RR, RG, RB, GR, GG, GB, BR, BG, BB \rbrace$\\
b) $\lbrace RG, RB, GR, GB, BR, BG \rbrace$\\

2.\\
a) $ A \cup B \cup C$\\
b) $ A - B - C$\\
c) $(A \cup B \cup C) - (A \cap B) - (A \cap C) - (B \cap C)$\\
d) $(A \cup B \cup C)^{c}$\\
e) $(A \cup B) - C$\\

3.\\
$P(\bigcup A_{i}) = \sum P(A_{i})$\\
$P(a, b) = P(a) + P(b) = 0.5$\\
$P(b, c) = P(b) + P(c) = 0.8$\\
$P(a, b) = P(a) + P(c) = 0.7$\\
$P(c) = P(b) + 0.2$\\
$2 \cdot P(b) + 0.2 = 0.8$\\\\
$P(\lbrace b \rbrace) = 0.3$\\
$P(\lbrace c \rbrace) = 0.5$\\
$P(\lbrace a \rbrace) = 0.2$\\ 
$P(\lbrace a, b \rbrace) = 0.5$\\ 
$P(\lbrace b, c \rbrace) = 0.8$\\ 
$P(\lbrace a, c \rbrace) = 0.7$\\
$P(\lbrace a, b, c \rbrace) = 1$\\  
$P(\lbrace \varnothing \rbrace) = 0$\\ 

4.\\
Se $B \subset A, P(A \cap B) = P(B) = 0.6$\\\\
Se $A \cup B = \Omega$\\
$P(A \cup B) = P(A) + P(B) - P(A \cap B)$\\
$P(\Omega) = 0.7 + 0.6 - P(A \cap B)$\\
$1 = 0.7 + 0.6 - P(A \cap B)$\\
$P(A \cap B) = 0.3$\\\\
$0.3 \leq P(A \cap B) \leq 0.6$\\

5.\\
a)\\
$P(A \cap B^{c}) = P(A) \cdot P(B^{c})$\\
$P(A - B) = P(A) \cdot (1 - P(B))$\\
$P(A - B) = P(A) - P(A \cap B) \hfill\blacksquare$\\\\
b) Simétrico à letra a, pois a interseção e a multiplicação são comutativas.\\\\
c) \\
$P(A^{c} \cap B^{c}) = P(A^{c}) \cdot P(B^{c})$\\
$P((A \cup B)^{c}) = (1 - P(A)) \cdot (1 - P(B))$\\
$1 - P(A \cup B) = 1 - P(A) - P(B) + P(A) \cdot P(B)$\\
$P(A \cup B) = P(A) + P(B) - P(A \cap B)$\\
Propriedade 7 da probabilidade. $\hfill\blacksquare$\\

6.\\
$P(A \cup B) = P(A) + P(B) - P(A \cap B)$\\
$0.7 = 0.4 + p - P(A \cap B)$\\
$0.3 =  p - P(A \cap B)$\\
$P(A \cap B) = p - 0.3$\\
$P(A \cap B) = P(A) \cdot P(B)$\\
$p - 0.3 = 0.4 \cdot p$\\
$0.6p = 0.3$\\
$p = 0.5$\\

7.\\
$\dfrac{P(C \cap D)}{P(D)} = 0.4$ e $\dfrac{P(C \cap D)}{P(C)} = 0.5$\\ 
$P(D) \cdot 0.4 = P(C \cap D) = P(C) \cdot 0.5$\\
$\dfrac{0.4}{0.5} = \dfrac{P(C)}{P(D)}$\\
$0.8 =  \dfrac{P(C)}{P(D)}$\\
$P(D) \geq P(C)$\\

8.

\Tree [. [.sabe [.acertou $P(x)=p$ ]
                [.errou $P(x)=0$ ]]
         [.não\ sabe [.acertou $P(x)=(1-p)(\frac{1}{m})$ ]
                     [.errou $P(x)=(1-p)(1-\frac{1}{m})$ ]]]\\
a)\\
$P(sabe|acertou) = \dfrac{P(sabe \cap acertou)}{P(acertou)}$\\
$P(sabe|acertou) = \dfrac{p}{p + \dfrac{1-p}{m}}$\\\\
b)\\
Quando o número de alternativas aumenta ao infinito, a chance de chutar uma questão e acertar diminui a zero.\\
$\displaystyle\lim_{m \to \infty} \dfrac{p}{p + \dfrac{1-p}{m}} = \dfrac{p}{p} = 1$\\\\
Quando a chance de saber a resposta uma questão diminui, as questão respondidas corretas serão de maioria as questões chutadas.\\
$\displaystyle\lim_{p \to 0} \dfrac{p}{p + \dfrac{1-p}{m}} = \dfrac{0}{0+\dfrac{1}{m}} = 0$











	
	
	
	
	
\end{document}