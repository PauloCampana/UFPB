\documentclass[12pt]{article}
\usepackage[a4paper, left=2cm, right=2cm, top=2cm, bottom=2cm]{geometry}
\usepackage{dsfont}
\usepackage[fleqn]{amsmath}
\usepackage{amssymb,amstext}
\usepackage{hyperref}
\begin{document}
\setcounter{secnumdepth}{0}
\newcommand{\p}[2]{\left\langle#1,\;#2\right\rangle}
\newcommand{\norma}[1]{\left\|\left(#1\right)\right\|}
\begin{center}
	UNIVERSIDADE FEDERAL DA PARAÍBA\\
	Introdução à Álgebra Linear\\
	Terceira Prova\\
	Paulo Ricardo Seganfredo Campana - 20210044220
\end{center}

\section{Questão 1.}
\subsection{i)}

Produto interno é uma função de $V \longrightarrow \mathds{R}$ que obedece 4 propriedades:\\
\[p1: \p{v}{v} \geq 0 \quad \forall v \text{ e somente 0 quando } v=0\]
\[p2: \p{u+v}{w} = \p{u}{w}+\p{v}{w}\]
\[p3: \p{u}{v} = \p{v}{u}\]
\[p4: \p{\alpha u}{v} = \alpha \p{u}{v} \quad \forall \alpha \in \mathds{R}\]

\subsection{ii)}

\[\p{u}{v} = x_{1}y_{1}-x_{1}y_{2}-x_{2}y_{1}+3x_{2}y_{2}\]
\[p1: \p{u}{u} = x_{1}^{2}-x_{1}x_{2}-x_{2}x_{1}+3x_{2}^{2} = x_{1}^{2}-2x_{2}x_{1}+3x_{2}^{2} = (x_{1}-x_{2})^{2}+2x_{2}^{2} \geq 0\]
\[\qquad \text{ pois são números reais ao quadrado e somente } 0 \text{ quando } u = (0,0)\]
\[p2: \p{u+v}{w} = (x_{1}+y_{1})z_{1}-(x_{1}+y_{1})z_{2}-(x_{2}+y_{2})z_{1}+3(x_{2}+y_{2})z_{2} =\]
\[\qquad = x_{1}z_{1}+y_{1}z_{1}-x_{1}z_{2}-y_{1}z_{2}-x_{2}z_{1}-y_{2}z_{1}+3x_{2}z_{2}+3y_{2}z_{2} =\]
\[\qquad = x_{1}z_{1}-x_{1}z_{2}-x_{2}z_{1}+3x_{2}z_{2}+y_{1}z_{1}-y_{1}z_{2}-y_{2}z_{1}+3y_{2}z_{2} = \p{u}{w}+\p{v}{w}\]
\[p3: \p{u}{v} = x_{1}y_{1}-x_{1}y_{2}-x_{2}y_{1}+3x_{2}y_{2} = y_{1}x_{1}-y_{1}x_{2}-y_{2}x_{1}+3y_{2}x_{2} = \p{v}{u}\]
\[p4: \p{\alpha u}{v} = \alpha x_{1}y_{1}-\alpha x_{1}y_{2}-\alpha x_{2}y_{1}+3\alpha x_{2}y_{2} =\]
\[\qquad \alpha(x_{1}y_{1}-x_{1}y_{2}-x_{2}y_{1}+3x_{2}y_{2}) = \alpha \p{u}{v}\]

\subsection{iii)}

\[\|(3,4)-(1,5)\| = \|(2,-1)\| = \sqrt{\p{(2,-1)}{(2,-1)}}\]
Com o produto interno usual:
\[\sqrt{\p{(2,-1)}{(2,-1)}} = \sqrt{4+1} = \sqrt{5}\]
Com o produto interno $x_{1}y_{1}-x_{1}y_{2}-x_{2}y_{1}+3x_{2}y_{2}$
\[\sqrt{\p{(2,-1)}{(2,-1)}} = \sqrt{4+4+3} = \sqrt{11}\]
São diferentes pois usam produtos interno diferentes.
\section{Questão 2.}

\[\|u+v\|^{2} - \|u-v\|^{2}\]
\[\p{u+v}{u+v} - \p{u-v}{u-v}\]
\[\p{u}{u} + 2\p{u}{v} + \p{v}{v} - \p{u}{u} + 2\p{u}{v} - \p{v}{v})\]
\[4\p{u}{v}\]
\[4\p{u}{v} = \|u+v\|^{2} - \|u-v\|^{2}\]

\section{Questão 3.}

\[\p{p}{q} = a_{0}b_{0}+a_{1}b_{1}+a_{2}b_{2}\]

\subsection{i)}

\[\p{x^{2}-2x+3}{3x-4} = 1\cdot0 + (-2)\cdot3 + 3\cdot(-4) = -18\]

\subsection{ii)}

\[\|x^{2}-2x+3\| = \sqrt{\p{x^{2}-2x+3}{x^{2}-2x+3}} = \sqrt{1\cdot1 + (-2)\cdot(-2) + 3\cdot3} = \sqrt{14}\]
\[\|3x-4\| = \sqrt{\p{3x-4}{3x-4}} = \sqrt{3\cdot3 + (-4)\cdot(-4)} = \sqrt{25} = 5\]

\subsection{iii)}

\[\|x^{2}-2x+3 + 3x-4\| = \|x^{2}+x-1\| = \sqrt{\p{x^{2}+x-1}{x^{2}+x-1}} =\]
\[= \sqrt{1\cdot1+1\cdot1+(-1)\cdot(-1)} = \sqrt{3}\]

\subsection{iv)}

\[\p{u}{v} = \cos\theta \|u\| \|v\|\]
\[\p{x^{2}-2x+3}{3x-4} = \cos\theta \|x^{2}-2x+3\| \|3x-4\|\]
\[-18 = \cos\theta \sqrt{14} \cdot 5\]
\[\cos\theta = \dfrac{-18}{5\sqrt{14}}\]
\[\theta = \arccos\left( \dfrac{-18}{5\sqrt{14}} \right)\]

\subsection{v)}

\[W = [3x-4] \qquad W^{\perp} = \lbrace (a_{2}x^{2}+a_{1}x+a_0) \in \mathds{P}_{2} ; \p{a_{2}x^{2}+a_{1}x+a_0}{3x-4} = 0 \rbrace\]
\[\p{a_{2}x^{2}+a_{1}x+a_0}{3x-4} = 3a_{1}-4a_{0} = 0\]
\[3a_{1}=4a_{0} \longrightarrow \dfrac{3}{4}a_{1} = a_{0} \text{ e } a_{2} \text{ livre}\]
\[p = \left( a_{2}x^{2} + a_{1}x + \dfrac{3}{4}a_{1} \right) = a_{2}(x^{2}) + a_{1} \left( x + \dfrac{3}{4} \right)\]
\[W^{\perp} = \left[ x^{2}, \left( x + \dfrac{3}{4} \right) \right]\]

\section{Questão 4.}
\subsection{i)}

\[v_{1}' = (1,1,1,1) \quad  u_{1} = \dfrac{(1,1,1,1)}{\norma{1,1,1,1}} = \dfrac{(1,1,1,1)}{2} = \left( \dfrac{1}{2}, \dfrac{1}{2},\dfrac{1}{2},\dfrac{1}{2} \right) \]
\[v_{2}' = (1,1,2,4) - \dfrac{\p{(1,1,2,4)}{(1,1,1,1)}}{\p{(1,1,1,1)}{(1,1,1,1)}}(1,1,1,1) = (1,1,2,4) - \dfrac{8}{4}(1,1,1,1) = (-1,-1,0,2)\]
\[u_{2} = \dfrac{(-1,-1,0,2)}{\norma{-1,-1,0,2}} = \dfrac{(-1,-1,0,2)}{\sqrt{6}} = \left( -\dfrac{1}{\sqrt{6}}, -\dfrac{1}{\sqrt{6}}, 0, \dfrac{2}{\sqrt{6}} \right)\]
\[v_{3}' = (1,2,-4,-3) - \dfrac{\p{(1,2,-4,-3)}{(-1,-1,0,2)}}{\p{(-1,-1,0,2)}{(-1,-1,0,2)}}(-1,-1,0,2) - \dfrac{\p{(1,2,-4,-3)}{(1,1,1,1)}}{\p{(1,1,1,1)}{(1,1,1,1)}}(1,1,1,1)\]
\[(1,2,-4,-3) - \dfrac{-9}{6}(-1,-1,0,2) - \dfrac{-4}{4}(1,1,1,1)\]
\[(2,3,-3,-2) + \left(-\dfrac{3}{2}, -\dfrac{3}{2}, 0, 3\right) = \left( \dfrac{1}{2}, \dfrac{3}{2},-3,1 \right)\]
\[u_{3} = \dfrac{\left( \dfrac{1}{2}, \dfrac{3}{2},-3,1 \right)}{\norma{\dfrac{1}{2}, \dfrac{3}{2},-3,1}} = \dfrac{\left( \dfrac{1}{2}, \dfrac{3}{2},-3,1 \right)}{\sqrt{\dfrac{25}{2}}} = \dfrac{\sqrt{2}}{5} \left( \dfrac{1}{2}, \dfrac{3}{2},-3,1 \right)  = \left( \dfrac{\sqrt{2}}{10}, \dfrac{3\sqrt{2}}{10}, \dfrac{-3\sqrt{2}}{5}, \dfrac{\sqrt{2}}{5} \right)\]
\[\beta = \left\lbrace \left( \dfrac{1}{2}, \dfrac{1}{2},\dfrac{1}{2},\dfrac{1}{2} \right), \left( -\dfrac{1}{\sqrt{6}}, -\dfrac{1}{\sqrt{6}}, 0, \dfrac{2}{\sqrt{6}} \right), \left( \dfrac{\sqrt{2}}{10}, \dfrac{3\sqrt{2}}{10}, \dfrac{-3\sqrt{2}}{5}, \dfrac{\sqrt{2}}{5} \right) \right\rbrace\]

\subsection{ii)}

\[\omega = (1,2,-6,\pi) = a\left( \dfrac{1}{2}, \dfrac{1}{2},\dfrac{1}{2},\dfrac{1}{2} \right) + b \left( -\dfrac{1}{\sqrt{6}}, -\dfrac{1}{\sqrt{6}}, 0, \dfrac{2}{\sqrt{6}} \right) + c \left( \dfrac{\sqrt{2}}{10}, \dfrac{3\sqrt{2}}{10}, \dfrac{-3\sqrt{2}}{5}, \dfrac{\sqrt{2}}{5} \right)\]
\[\left\{ \begin{array}{@{}l@{}}
	\dfrac{a}{2} - \dfrac{b}{\sqrt{6}} + \dfrac{c\sqrt{2}}{10} = 1\\
	\dfrac{a}{2} - \dfrac{b}{\sqrt{6}} + \dfrac{3c\sqrt{2}}{10} = 2\\
	\dfrac{a}{2} + 0                   + \dfrac{-3c\sqrt{2}}{5} = -6\\
	\dfrac{a}{2} +\dfrac{2b}{\sqrt{6}} + \dfrac{c\sqrt{2}}{5} = \pi\\
\end{array} \right. \longrightarrow
\left\{ \begin{array}{@{}l@{}}
	0            - 0                   + \dfrac{2c\sqrt{2}}{10} = 1 \longrightarrow c = \dfrac{5}{\sqrt{2}}\\
	\dfrac{a}{2} + 0                   - 3 = -6 \longrightarrow \dfrac{a}{2} = -3 \longrightarrow a = -6\\
	-3 -\dfrac{b}{\sqrt{6}} + \dfrac{1}{2} = 1 \longrightarrow \dfrac{b}{\sqrt{6}} = -\dfrac{7}{2} \longrightarrow b = -\dfrac{7\sqrt{6}}{2}\\
	\dfrac{a}{2} +\dfrac{2b}{\sqrt{6}} + \dfrac{c\sqrt{2}}{5} = \pi\\
\end{array} \right.\]
Porém o conjunto $\left\lbrace -6, -\dfrac{7\sqrt{6}}{2}, \dfrac{5}{\sqrt{2}} \right\rbrace$ não é solução para a quarta equação $\dfrac{a}{2} +\dfrac{2b}{\sqrt{6}} + \dfrac{c\sqrt{2}}{5} = \pi$
O sistema não tem solução, portanto $\omega = (1,2,-6,\pi)$ não pode ser escrito como combinação linear de $\beta$.















\subsection{iii)}

\[W=[(1,1,1,1),(1,1,2,4),(1,2,-4,-3)] \qquad W^{\perp} = \lbrace (x,y,z,t) \in \mathds{R}^{4} ; \p{(x,y,z,t)}{w} = 0\rbrace\]
\[\p{(x,y,z,t)}{(1,1,1,1)}=x+y+z+t=0\]
\[\p{(x,y,z,t)}{(1,1,2,4)}=x+y+2z+4t=0\]
\[\p{(x,y,z,t)}{(1,2,-4,-3}=x+2y+-4z-3t=0\]
\[\left\{ \begin{array}{@{}l@{}}
	x+y+z+t=0\\
	x+y+2z+4t=0 \longrightarrow z+3t=0 \longrightarrow z=-3t\\
	x+2y-4z-3t=0 \longrightarrow y-5z-4t=0 \longrightarrow y+11t=0 \longrightarrow y=-11t
\end{array} \right.\]
\[(x,y,z,t) = (13t, -11t, -3t, t) = t(13,-11,-3,1)\]
\[W^{\perp} = [(13,-11,-3,1)]\]

\end{document}