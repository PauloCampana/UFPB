\documentclass[12pt]{article}
\usepackage[utf8]{inputenc}
\usepackage[
a4paper,		% papel A4
left=2cm,		% margem esquerda
right=2cm,		% margem direita
top=2cm,		% margem superior
bottom=2cm		% margem inferior
]{geometry}
\usepackage[T1]{fontenc}	
\usepackage{array,latexsym,amssymb}
\usepackage{amsmath,amsfonts,amssymb,amsthm,mathabx,amstext}
\usepackage{dsfont}	% Conjuntos: $\mathds{N, Z, Q, R, C}$
\usepackage{graphicx}
\begin{document}
\begin{center}
	UNIVERSIDADE FEDERAL DA PARAÍBA\\
	Introdução à Álgebra Linear\\
	Segunda Lista de Exercícios\\
	Paulo Ricardo Seganfredo Campana\\
\end{center}
	
\noindent Questão 1.\\

\noindent a)\\

Sendo $u=(x_{1},y_{1})$ e $v=(x_{2},y_{2})$\\

$T(u+v)=T(x_{1}+x_{2},y_{1}+y_{2})=(x_{1}+x_{2}+2y_{1}+2y_{2},3x_{1}+3x_{2}-y_{1}-y_{2})$\\

$(x_{1}+2y_{1},3x_{1}-y_{1})+(x_{2}+2y_{2},3x_{2}-y_{2})=T(x_{1},y_{1})+T(x_{2},y_{2})=T(u)+T(v)$\\

$T(\alpha u)=T(\alpha x_{1}, \alpha y_{1})=(\alpha x_{1}+2\alpha y_{1},3\alpha x_{1}-\alpha y_{1})=\alpha(x_{1}+2y_{1},3x_{1}-y_{1})$\\

$\alpha T(x_{1},y_{1})=\alpha T(u)\hfill\blacksquare$\\

\noindent b)\\

Sendo $u=(x_{1},y_{1},z_{1})$, $v=(x_{2},y_{2},z_{2})$\\

$T(u+v)=T(x_{1}+x_{2},y_{1}+y_{2},z_{1},z_{2})=(2x_{1}+2x_{2}-y_{1}-y_{2}+z_{1}+z_{2},y_{1}+y_{2}-4z_{1}-4z_{2})$\\
	
$(2x_{1}-y_{1}+z_{1},y_{1}-4z_{1})+(2x_{2}-y_{2}+z_{2},y_{2}-4z_{2})=T(x_{1},y_{1},z_{1})+T(x_{2},y_{2},z_{2})=T(u)+T(v)$\\

$T(\alpha u)=T(\alpha x_{1},\alpha y_{1},\alpha z_{1})=(2\alpha x_{1}-\alpha y_{1}+\alpha z_{1},\alpha y_{1}-4\alpha z_{1})=\alpha(2x_{1}-y_{1}+z_{1},y_{1}-4z_{1})$\\

$\alpha T(x_{1},y_{1},z_{1})=\alpha T(u)\hfill\blacksquare$\\

\noindent c)\\

$T(A+A')=(A+A')\cdot B=AB+A'B=T(A)+T(A')$\\

$T(\alpha A)=(\alpha A)\cdot B=\alpha(AB)=\alpha T(A)\hfill\blacksquare$\\

\noindent d)\\

$T(A+B)=tr(A+B)=(a_{11}+b_{11}+a_{22}+b_{22}+...+a_{nn}+b_{nn})$\\

$(a_{11}+a_{22}+...+a_{nn})+(b_{11}+b_{22}+...+b_{nn})=tr(A)+tr(B)=T(A)+T(B)$\\

$T(\alpha A)=tr(\alpha A)=(\alpha a_{11}+\alpha a_{22}+...+\alpha a_{nn}) = \alpha(a_{11}+a_{22}+...+a_{nn})=\alpha tr(A)=\alpha T(A)\hfill\blacksquare$\\

\noindent e)\\

Não é transformação pois $T(0)=T(0+0x+0x^{2})=(1,x,x^{2})$ ou seja, $T(0)\neq0\hfill\blacksquare$\\

\noindent Questão 2.\\

$T(1,1,1)=(2,-1,4)\qquad T(1,1,0)=(3,0,1)\qquad T(1,0,0)=(-1,5,1)$\\

$(x,y,z)=a(1,1,1)+b(1,1,0)+c(1,0,0)=(a+b+c,a+b,a)$\\

$\left\{\begin{array}{@{}l@{}}
	a+b+c=x\\
	a+b=y\\
	a=z
\end{array}\right.$\qquad\qquad
$a=z,\qquad b=y-z,\qquad c=x-y$\\

$T(x,y,z)=zT(1,1,1)+(y-z)T(1,1,0)+(x-y)T(1,0,0)$\\

$z(2,-1,4)+(y-z)(3,0,1)+(x-y)T(-1,5,1)$\\

$(2z,-z,4z)+(3y-3z,0,y-z)+(-x+y,5x-5y,x-y)=(-x+4y-z,5x-5y-z,x+3z)$\\

$T(2,4,-1)=(15,-9,-1)$\\

\noindent Questão 3.\\

$T(2v_{1}-4v_{2}+5v_{3})=T(2v_{1})+T(-4v_{2})+T(5v_{3})=2T(v_{1})-4T(v_{2})+5T(v_{3})$\\

$2(1,-1,2)-4(0,3,2)+5(-3,1,2)=(2,-2,4)+(0,-12,-8)+(-15,5,10)=(-13,-9,6)$\\

\noindent Questão 4.\\

$n^n$ pois a transformação $T$ pode levar qualquer $v_{i}$ em qualquer $v_{j}$ incluindo $i=j$, ou seja, cada vetor $v_{i}$ pode ser mapeado a $n$ diferentes vetores, e existem $n$ vetores $v_{i}$, portanto existem $n^n$ transformações.\\

\noindent Questão 5.\\

\noindent a)\\

$Im(T)=\lbrace(a,b,c);a=(4x+y-2z-3t),b=(2x+y+z-4t),c=(6x-9z+9t)\rbrace$\\

$\lbrace(4x+y-2z-3t,2x+y+z-4t,6x-9z+9t)\forall x,y,z,t \in \mathds{R}\rbrace$\\

$\lbrace x(4,2,6)+y(1,1,0)+z(-2,1,-9)+t(-3,-4,9)\forall x,y,z,t \in \mathds{R}\rbrace$\\

$[(4,2,6),(1,1,0),(-2,1,-9),(-3,-4,9)]$\\

$\begin{bmatrix}
	4 & 2 & 6 \\
	1 & 1 & 0 \\
	-1 & 1 & -9 \\
	-3 & -4 & 9
\end{bmatrix}$\qquad
$\begin{bmatrix}
	4 & 2 & 6 \\
	0 & 2 & -9 \\
	0 & -7 & 36 \\
	0 & -1 & 9
\end{bmatrix}$\qquad
$\begin{bmatrix}
	4 & 2 & 6 \\
	0 & -1 & 9 \\
	0 & 0 & 9 \\
	0 & 0 & -27
\end{bmatrix}$\qquad
$\begin{bmatrix}
	4 & 0 & 24 \\
	0 & 1 & -9 \\
	0 & 0 & 9 \\
	0 & 0 & 0
\end{bmatrix}$\qquad
$\begin{bmatrix}
	1 & 0 & 0 \\
	0 & 1 & 0 \\
	0 & 0 & 1 \\
	0 & 0 & 0
\end{bmatrix}$\\

$Im(T)=[(1,0,0),(0,1,0),(0,0,1)]=\mathds{R}^{3}$, os três vetores estão dentro de $Im(T)$\\

\noindent b)\\

$N(T)\Longrightarrow(4x+y-2z-3t,2x+y+z-4t,6x-9z+9t)=0$\\

$\begin{bmatrix}
	4 & 1 & -2 & -3 & 0 \\
	2 & 1 & 1 & -4 & 0 \\
	6 & 0 & -9 & 9 & 0
\end{bmatrix}$\qquad
$\begin{bmatrix}
	2 & 1 & 1 & -4 & 0 \\
	0 & 1 & 4 & -5 & 0 \\
	0 & -1 & -4 & 9 & 0
\end{bmatrix}$\qquad
$\begin{bmatrix}
	2 & 1 & 1 & -4 & 0 \\
	0 & 1 & 4 & -5 & 0 \\
	0 & 0 & 0 & 9 & 0
\end{bmatrix}\Longrightarrow t=0$\\

$\begin{bmatrix}
	2 & 1 & 1 & 0 \\
	0 & 1 & 4 & 0
\end{bmatrix}$\qquad
$\begin{bmatrix}
	2 & 0 & -3 & 0 \\
	0 & 1 & 4 & 0
\end{bmatrix}$\qquad
$\begin{bmatrix}
	8 & 3 & 0 & 0 \\
	0 & 1 & 4 & 0
\end{bmatrix}$\qquad
$\left\{\begin{array}{@{}l@{}}
	8x+3y=0\\
	y+4z=0
\end{array}\right.\Longrightarrow [(8\alpha,-3\alpha,12\alpha,0)]$\\

$dimN+dimIm=dim\mathds{R}^{4}$\qquad\qquad
$1+3=4$\quad\checkmark

\noindent c)\\

Apenas $u=(3,-8,2,0)$ pois $(12-8-4-0,6-8+2+0,18-18+0)=(0,0,0)$\\
	
\noindent Questão 6.\\

$T(P_{2})=T(a+bx+cx^{2})=x(a+bx+cx^{2})=ax+bx^{2}+cx^{3}$\\

$N(T)\Longrightarrow ax+bx^{2}+cx^{3}=0$ possui única solução trivial $(a,b,c)=(0,0,0)$\qquad $N(T)=0$\\

$Im(T)\Longrightarrow \lbrace ax+bx^{2}+cx^{3}\rbrace=\lbrace a(x)+b(x^{2})+c(x^{3})\rbrace=[(x,x^{2},x^{3})]$\\
	
\noindent{\Huge Boldrini - Capítulo 5}\\

\noindent Questão 3.\\

$(x,y,z)=a(1,0,0)+b(0,1,0)+c(0,0,1)=(a,b,c)\qquad\qquad x=a,\quad y=b,\quad z=c$\\

$T(x,y,z)=xT(1,0,0)+yT(0,1,0)+zT(0,0,1)=$\\

$x(2,0)+y(1,1)+z(0,-1)=(2x+y,y-z)=(3,2)$\\

$\left\{\begin{array}{@{}l@{}}2x+y=3\\y-z=2\end{array}\right.\qquad\qquad z=1-2x,\quad y=3-2x\qquad\qquad v=(x, 3-2x, 1-2x)$\\

\noindent Questão 4.\\

\noindent a)\\

$(x,y)=a(1,1)+b(0,-2)=(a,a-2b)\qquad\qquad x=a,\quad y=x-2b,\quad b = \dfrac{x-y}{2}$\\

$T(x,y)=xT(1,1)+(\dfrac{x-y}{2})T(0,-2)=x(3,2,1)+(\dfrac{x-y}{2})(0,1,0)=(3x,\dfrac{5x-y}{2},x)$\\

\noindent b)\\
	
$T(1,0)=\left(3,\dfrac{5}{2},1\right)\qquad\qquad T(0,1)=\left(0,-\dfrac{1}{2},0\right)$\\

\noindent c)\\

$(x,y,z)=a(3,2,1)+b(0,1,0)+c(0,0,1)=(3a,2a+b,a+c)$\\

$a=\dfrac{x}{3},\quad b=y-\dfrac{2x}{3},\quad c=z-\dfrac{x}{3}$\\

$S(x,y,z)=\dfrac{x}{3}S(3,2,1)+(y-\dfrac{2x}{3})S(0,1,0)+(z-\dfrac{x}{3})S(0,0,1)$\\

$\dfrac{x}{3}(1,1)+(y-\dfrac{2x}{3})(0,-2)+(z-\dfrac{x}{3})(0,0)=(\dfrac{x}{3},-2y+\dfrac{5x}{3})$\\

\noindent d)\\

$P(x,y)=S(T(x,y))=S(3x,\dfrac{5x-y}{2},x)=(x,y)$\\

\noindent Questão 5.\\

$T(x,y)=(y,x)$\\

$\begin{bmatrix}0&1\\1&0\end{bmatrix}$\\

\noindent Questão 10.\\

$R=S(T)\qquad T=S^{-1}(R)$\\

$S^{-1}=\begin{bmatrix}4&2&3\\2&1&1\\-7&-3&-5\end{bmatrix}\qquad\qquad T=\begin{bmatrix}8&0&9\\4&0&4\\-13&2&-15\end{bmatrix}$\\

\noindent Questão 11.\\

\noindent a)\\

$(x,y)=a(1,-1)+b(0,2)=(a,2b-a)\qquad\qquad a=x,\quad b=\dfrac{x+y}{2}$\\

$[T]_{\beta}=[T]_{\beta}^{\alpha}[V]_{\alpha}=\begin{bmatrix}1&0\\1&1\\0&-1\end{bmatrix}\begin{bmatrix}x\\\dfrac{x+y}{2}\end{bmatrix}=\begin{bmatrix}x\\\dfrac{3x+y}{2}\\\dfrac{-x-y}{2}\end{bmatrix}$\\
	
$T=x(1,0,-1)+(\dfrac{3x+y}{2})(0,1,2)+(\dfrac{-x-y}{2})(1,2,0)=(\dfrac{x-y}{2},\dfrac{x-y}{2},2x-y)$\\

\noindent b)\\

$S(1,-1)=(-2,2,1)=a(1,0,-1)+b(0,1,2)+c(1,2,0)=(a+c,b+2c,2b-a)$\\

$\left\{\begin{array}{@{}l@{}}a+c=-2\\b+2c=2\\2b-a=1\end{array}\right.\qquad\qquad a=\dfrac{-11}{3},\quad b=\dfrac{-4}{3},\quad c=\dfrac{5}{3}$\\
	
$S(1,-1)=(4,-2,0)=a(1,0,-1)+b(0,1,2)+c(1,2,0)=(a+c,b+2c,2b-a)$\\

$\left\{\begin{array}{@{}l@{}}a+c=4\\b+2c=-2\\2b-a=0\end{array}\right.\qquad\qquad a=\dfrac{20}{3},\quad b=\dfrac{10}{3},\quad c=\dfrac{-8}{3}$\\

$[S]^{\alpha}_{\beta}=\begin{bmatrix}\dfrac{-11}{3}&\dfrac{20}{3}\\\dfrac{-4}{3}&\dfrac{10}{3}\\\dfrac{5}{3}&\dfrac{-8}{3}\end{bmatrix}$\\
	
\noindent c)\\

$[T]_{\gamma}=[T]_{\gamma}^{\alpha}[T]_{\alpha}=\begin{bmatrix}1&0\\0&0\\0&1\end{bmatrix}\begin{bmatrix}x\\\dfrac{x+y}{2}\end{bmatrix}=\begin{bmatrix}x\\0\\\dfrac{x+y}{2}\end{bmatrix}$\\

$\gamma=\lbrace(1,1,1),(x,y,z),(-1,-1,2)\rbrace$\\

\noindent Questão 14.\\

\noindent a)\\

$T(1,0,0,0)=(1,0)\quad T(0,1,0,0)=(0,1)\quad T(0,0,1,0)=(0,1)\quad T(0,0,0,1)=(1,0)$\\

$(1,0)=1(1,0)+0(0,1)\quad(0,1)=0(1,0)+1(0,1)$\\

$[T]_{\alpha}^{\beta}=\begin{bmatrix}1&0&0&1\\0&1&1&0\end{bmatrix}$\\

\noindent b)\\
	
$S(x,y)=\begin{bmatrix}2x+y&x-y\\-x&y\end{bmatrix}=\begin{bmatrix}1&0\\0&1\end{bmatrix}$\\

$x=y\quad x=0\quad y=1\qquad\qquad$ impossível.\\

\noindent Questão 19.\\

\noindent a)\quad$z=0\quad x=y\quad [(1,1,0)]$\\

\noindent b)\quad$dimN(T)+dimIm(T)=dim\mathds{R}^{3}\qquad 1+dimIm(T)=3\qquad dimIm(T)=2$\\
	
\noindent c)\quad não pois $Im(T)\neq\mathds{R}^{3}$\\

\noindent d)\quad Núcleo é uma reta no $\mathds{R}^{3}$ e Imagem é um plano no $\mathds{R}^{3}$\\

\noindent Questão 20.\\

\noindent a)\quad$T(x,y,z)=(y,z)$\\

\noindent b)\quad não existe pois $dim\mathds{R}^{3}>dim\mathds{R}^{2}$\\

\noindent c)\quad$L(x,y,z)=(0,0)$\\

\noindent d)\quad$M(x,y)=(0,0)$\\

\noindent e)\quad$H(x,y,z)=(0,0,0)$\\

\noindent Questão 22.\\

$D(a_{3}x^{3}+a_{2}x^{2}+a_{1}x+a_{0})=(6a_{3}x+2a_{2})$\\

$D(a+b)=(6(a_{3}+b_{3})x+2(a_{2}+b_{2}))=(6a_{3}x+2a_{2})+(6b_{3}x+2b_{2})=D(a)+D(b)$\\

$D(\alpha a)=(6\alpha a_{3}x+2\alpha a_{2})=\alpha(6a_{3}x+2a_{2})=\alpha D(a)$\\

$N(D)=(0+0+a_{1}x+a_{0})=\lbrace x,1\rbrace$\\

\end{document}