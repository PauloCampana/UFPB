\documentclass[12pt]{article}
\usepackage[utf8]{inputenc}
\usepackage[T1]{fontenc}	
\usepackage{array,latexsym}
\usepackage{amsmath,amsfonts,amssymb,amsthm,mathabx,amstext}
\usepackage{dsfont}	% Conjuntos: $\mathds{N, Z, Q, R, C}$
\usepackage{graphicx}
\begin{document}
\pagestyle{empty}

\begin{center}
	UNIVERSIDADE FEDERAL DA PARAÍBA\\
	Introdução à Álgebra Linear\\
	Primeira Lista de Exercícios\\
	Paulo Ricardo Seganfredo Campana\\
\end{center}
	
Questão I\\
1.) Não, visto que não satisfaz a propriedade 6 para todos os vetores em $\mathds{R}^{3}$\\
$(\alpha + \beta)u = \alpha u + \beta u$\\
$(\alpha + \beta)(x,y,z) = \alpha (x,y,z) + \beta (x,y,z)$\\
$((\alpha + \beta)x,y,z) = (\alpha x,y,z) + (\beta x,y,z)$\\
$((\alpha + \beta)x,y,z) = (\alpha x + \beta x, 2y, 2z)$\\
$y \neq 2y, z \neq 2z\hfill\blacksquare$\\

2.) Não, pois não satisfaz a propriedade 8 para todos os vetores em $\mathds{R}^{3}$\\
$1u = u$\\
$1(x,y,z) = (x,y,z)$\\
$(0,0,0) = (x,y,z)\hfill\blacksquare$\\

3.)  Não, pois não satisfaz a propriedade 8 para todos os vetores em $\mathds{R}^{2}$\\
$1u = u$\\
$1(x,y) = (x,y)$\\
$(2x,2y)=(x,y)\hfill\blacksquare$\\

4.) Sim\\

5.) Não, pois não satisfaz a propriedade 3 para todos os vetores em $\mathds{R}^{2}$\\
$\exists0\in V, \text{tal que }u+0=u$\\
$(x,y)+(0,0)=(x,y)$\\
$(x+0+1,y+0+1)=(x,y)$\\
$(x+1,y+1)=(x,y)\hfill\blacksquare$\\

6.) Não, pois não satisfaz a propriedade 3 para todas as matrizes na forma 
$\begin{bmatrix}
	a & 1 \\
	1 & b
\end{bmatrix}$.\\

$\exists0\in V, \text{tal que }u+0=u$\\
$\begin{bmatrix}
	a & 1 \\
	1 & b
\end{bmatrix} + 
\begin{bmatrix}
	x & 1 \\
	1 & y
\end{bmatrix} = 
\begin{bmatrix}
	a & 1 \\
	1 & b
\end{bmatrix}$\\
$\begin{bmatrix}
	a+x & 2 \\
	2 & b+y
\end{bmatrix} = 
\begin{bmatrix}
	a & 1 \\
	1 & b
\end{bmatrix}\hfill\blacksquare$\\

7.) Não, pois não satisfaz a propriedade 3 para todas as funções em $F = \lbrace f: \mathds{R} \rightarrow \mathds{R}; f(0) = 1 \rbrace$.\\
$\exists0\in V, \text{tal que }u+0=u$\\
$(f+0)(x) = f(x) + f(x)$\\
$f(x) = 2f(x)\hfill\blacksquare$\\

8.) Sim\\

Questão II\\
Não, pois pela segunda propriedade de Espaço Vetorial:\\
$\forall \alpha \in \mathds{R}, \forall u \in V, \alpha u \in V$\\
Indica que, se existe um r vetor não nulo, irá existir infinitos vetores r multiplicados por uma constante.\\
Portanto, os dois vetores desse Espaço Vetorial terão de ser nulos, porém não se pode ter dois vetores iguais em um Espaço Vetorial, então V terá apenas um elemento.\\

Questão III\\
Sim, LUA será um vetor nulo, obedece as 8 propriedades, e o conjunto formado por um vetor nulo é um Espaço Vetorial.\\
p1: $(l+l)+l=l+(l+l)\Longrightarrow l+l=l+l\Longrightarrow l=l$\\
p2: $l+l=l+l\Longrightarrow l=l$\\
p3: $l=0\Longrightarrow l+l=l\Longrightarrow l=l$\\
p4: $-u=l\Longrightarrow l+l=0=l\Longrightarrow l=l$\\
p5: $(\alpha\beta)l = \alpha(\beta l) \Longrightarrow l = \alpha l \Longrightarrow l=l$\\
p6: $(\alpha+\beta)l = \alpha l +\beta l \Longrightarrow l = l+l \Longrightarrow l=l$\\
p7: $\alpha(l+l)=\alpha l+\alpha l\Longrightarrow\alpha l=l+l\Longrightarrow l=l$\\
p8: $1\cdot l = l \Longrightarrow l = l\hfill\blacksquare$\\

Questão IV\\
1. Não, pois a reta $y = a + z$ não necessariamente passa pela origem.\\
2. Não, pois não obedece a segunda propriedade de Subespaço, já que se multiplicado por um $\alpha$ não inteiro estará fora so Subespaço.
\par $\alpha u \in W, \forall u \in W, \alpha \in \mathds{R}$\\
3. Sim.\\

Questão V\\
Alternativa 3, pois:\\
$(0, 0, 0) = 0\cdot(2, 1, 4) + 0\cdot(3, 2, 5)$\\

Questão VI\\
1.)\\
\begin{equation}
\left\{\begin{array}{@{}l@{}}
	2x+y+3z=-9\\
	x-y+2z=-7\\
	4x+3y+5z=-15
\end{array}\right. \Longrightarrow
\left\{\begin{array}{@{}l@{}}
	3x+5z=-16\\
	7x+11z=-36
\end{array}\right. \Longrightarrow
\left\{\begin{array}{@{}l@{}}
	-21x-35z=112\\
	21x+33z=-108
\end{array}\right.
\end{equation}\\
$z = -2, x = -2, y = 1$\\\\
$u = (-9, -7, -15) = -2(2, 1, 4) + 1(1, -1, 3) -2(3, 2, 5)$\\
2.)\\
\begin{equation}
\left\{\begin{array}{@{}l@{}}
	2x+y+3z=-9\\
	x-y+2z=-7\\
	4x+3y+6z=-15
\end{array}\right. \Longrightarrow
\left\{\begin{array}{@{}l@{}}
	3x+5z=-16\\
	7x+12z=-36
\end{array}\right. \Longrightarrow
\left\{\begin{array}{@{}l@{}}
	-21x-35z=112\\
	21x+36z=-108
\end{array}\right.
\end{equation}\\
$z = 4, x = -12, y = 3$\\\\
$u = -9 -7x -15x{2} = -12(2 + x + 4x^{2}) + 3(1 - x + 3x^{2}) + 4(3 + 2x + 6x^{2})$\\

{\Huge Exercicios do Boldrini}\\

1.\\
a) Vetor nulo: $(0,0,\cdots,0)$, $-(x_{1},x_{2},\cdots,x_{n})$ é o vetor oposto\\
b) $\begin{bmatrix}0&0\\0&0\end{bmatrix}$ e 
$\begin{bmatrix}-x_{1}&-x_{2}\\-x_{3}&-x_{4}\end{bmatrix}$\\

3.\\
a) Sim, $a\begin{bmatrix}1&0\\0&0\end{bmatrix} + b\begin{bmatrix}0&1\\1&0\end{bmatrix} +
		 d\begin{bmatrix}0&0\\0&1\end{bmatrix}$.\\
b) Não.\\

5.\\
a) Sim, $dim = n$, sem perda de generalização para o caso 2x2: $\begin{bmatrix}1&0\\0&0\end{bmatrix}, \begin{bmatrix}0&0\\0&1\end{bmatrix}$\\
b) Sim, $dim = 1$, $\begin{bmatrix}1&0&\cdots&0\\0&1&\cdots&0\\\vdots&\vdots&\ddots&\vdots\\0&0&\cdots&1\end{bmatrix}$\\
c) Sim, $dim = 2$,
$\begin{bmatrix}1&1\\1&0\end{bmatrix}, \begin{bmatrix}0&1\\0&1\end{bmatrix}$\\
d) Sim, $dim = 1$, $(1,1,\cdots,1)$\\
e) Não, não terá vetor nulo\\
f) Não, não terá vetor nulo\\
g) Sim, $dim = 1$, $(1,2,3)$\\

7.\\
a) Sim, se $a = 0$ e $b = -1$\\
b) Não, pois o terceiro elemento tem de ser 0 para ser parte deste subespaço.\\

9.\\
São LI pois a única solução do sistema é a trivial ($a=b=c=d=0$) e geram o espaço através de:\\
$\begin{bmatrix}a&b\\c&d\end{bmatrix} = 
a\begin{bmatrix}1&0\\0&0\end{bmatrix} + 
b\begin{bmatrix}0&1\\0&0\end{bmatrix} +
c\begin{bmatrix}0&0\\1&0\end{bmatrix} +
d\begin{bmatrix}0&0\\0&1\end{bmatrix}$\\

11.\\
$\left\{\begin{array}{@{}l@{}}
	x-y+z=1\\
	x+y=0\\
	x-z=0
\end{array}\right. \Longrightarrow z=x,\quad y = -x$\\
$x+x+x=1 \Longrightarrow x = \dfrac{1}{3},\quad y = -\dfrac{1}{3},\quad z = \dfrac{1}{3}$\\

13.\\
Se estes 4 vetores são LI, irão gerar uma base em $P^{3}$\\
$a(1-t^{3})+b(1-t)^{2}+c(1-t)+d=0$\\
$a-at^{3}+b-bt^{2}-2bt+c-ct+d=0$\\
$-at^{3}+bt^{2}-(2b+c)t+(a+b+c+d)=0$\\
Só assume solução $a=b=c=d=0$, portanto são LI, formam base, e geram $P^{3}$\\

15.\\
A terceira matriz é combinação linear dos dois primeiros, para as outras matrizes gerarem o subespaço W, basta serem LI.\\
$\left\{\begin{array}{@{}l@{}}
	x+y+z=0\\
	-5x+y-7z=0\\
	-4x-y-5z=0\\
	2x+5y+z=0
\end{array}\right. \Longrightarrow x=-4y,\quad z=3y$\\
$-4y\begin{bmatrix}1&-5\\-4&2\end{bmatrix}+
   y\begin{bmatrix}1&1\\-1&5\end{bmatrix}+
  3y\begin{bmatrix}1&-7\\-5&1\end{bmatrix}=0$\\
$ \begin{bmatrix}1&1\\-1&5\end{bmatrix}=
-3\begin{bmatrix}1&-7\\-5&1\end{bmatrix}+
 4\begin{bmatrix}1&-5\\-4&2\end{bmatrix}$\\
Esta segunda matriz também é combinação linear, a primeira e a quarta são LI, portanto geram W, $dimW=2$.\\

17.\\
a) Sim, pois B foi obtido pela redução da matriz A.\\
b) Sim, pois seguirão a forma padrão de uma matriz diagonal de vetores LI que geram um subespaço.\\
$\begin{bmatrix}a_{11}&0&\cdots&0\\0&a_{22}&\cdots&0\\\vdots&\vdots&\ddots&\vdots\\0&0&\cdots&a_{mn}\end{bmatrix}$\\

19.\\
Porquê esses três vetores são LI, e três vetores LI são suficientes para gerar $\mathds{R}^{3}$\\
$\left\{\begin{array}{@{}l@{}}
	x+y=0\\
	-y+z=0\\
	x+y+z=0
\end{array}\right. \Longrightarrow y=z$\\
$\left\{\begin{array}{@{}l@{}}
	x+y=0\\
	x+2y=0
\end{array}\right. \Longrightarrow y=0,\quad x=0,\quad z=0$, São LI.\\

21.\\
a) $a=b=c=0$ para que estes vetores sejam LI.\\
b) $x=(-\dfrac{5}{3},0,0),\quad y=(0,\dfrac{7}{3},0),\quad z=(0,0,1)$\\
c) Quanto maior a dimensão de W, mais variáveis livres o sistema terá.\\

23.\\
$w_{1}+w_{2}\in W_{1}+W_{2}$\\
$w_{1}'+w_{2}'\in W_{1}+W_{2}$\\
$u+v = w_{1}+w_{2}+w_{1}'+w_{2}'$\\
$w_{1}+w_{2}+w_{1}'+w_{2}'\in W_{1}+W_{2}$\\
$u+v\in W_{1}+W_{2}\hfill\blacksquare$\\\\
$ku=kw_{1}+kw_{2}$\\
$kw_{1}+kw_{2}\in W_{1}+W_{2}$\\
$ku\in W_{1}+W_{2}\hfill\blacksquare$\\

25.\\
a)\\
$\left\{\begin{array}{@{}l@{}}
	x+y=0\\
	z-t=0\\
	x-y-z+t=0
\end{array}\right. \Longrightarrow y = -x,\quad z=t,\quad x+x-z+z=0$\\
$x=y=0,\quad z=t \in \mathds{R}$\\
$W_{1}\cap W_{2} = \lbrace(0,0,z,z): z \in \mathds{R} \rbrace$\\
b) exemplo de base: $(0,0,22,22)$.\\
c) Feito em sala por escalonamento\\
$W_{1}+ W_{2} = [(1,-1,0,0),(0,1,1,-1),(0,0,-1,2),(0,0,0,3)]$\\
d) Não, pois $W_{1}\cap W_{2} \neq (0,0,0,0)$\\
e) Sim, pois são quatro vetores LI, que pelas propriedades, geram $\mathds{R}^{4}$\\

27. a)\\
$(x,y,-x-2y) = x(1,0,-1) + y(0,1,-2)$\\
$V_{1} = [(1,0,-1),(0,1,-2)]$\\
$V_{2} = [(0,0,1)]$\\
b) A soma nunca será direta, pois dois subespaços de dimensão dois terão 4 vetores LI, porém $\mathds{R}^{3}$ necessita de 3 e somente 3 vetores LI para ser gerado.\\
$V_{1} = [(1,0,0),(0,1,0)]$\\
$V_{2} = [(0,1,0),(0,0,1)]$\\

29.\\
a) i) $\begin{bmatrix}-1&1\\1&1\end{bmatrix}$\\
$(-1,1)=a(1,0)+b(0,1) \Longrightarrow a=-1,\quad b=1$\\
$(1,1)=c(1,0)+d(0,1) \Longrightarrow c=d=1$\\
a) ii) $\begin{bmatrix}-\frac{1}{2}&\frac{1}{2}\\\frac{1}{2}&\frac{1}{2}\end{bmatrix}$\\
$(1,0)=a(-1,1)+b(1,1) \Longrightarrow a = -\frac{1}{2},\quad b=\frac{1}{2}$\\
$(0,1)=c(-1,1)+d(1,1) \Longrightarrow c=d=\frac{1}{2}$\\
a) iii) $\begin{bmatrix}\frac{\sqrt{3}}{6}&\frac{1}{2}\\\frac{\sqrt{3}}{6}&-\frac{1}{2}\end{bmatrix}$\\
$(1,0)=a(\sqrt{3},1)+b(\sqrt{3},-1) \Longrightarrow a=b=\frac{\sqrt{3}}{6}$\\
$(0,1)=c(\sqrt{3},1)+d(\sqrt{3},-1) \Longrightarrow c=\frac{1}{2},\quad d=-\frac{1}{2}$\\
a) iv) $\begin{bmatrix}\frac{1}{2}&0\\0&\frac{1}{2}\end{bmatrix}$\\
$(1,0)=a(2,0)+b(0,2) \Longrightarrow a=\frac{1}{2},\quad b=0$\\
$(0,1)=c(2,0)+d(0,2) \Longrightarrow c=0,\quad d=\frac{1}{2}$\\
b) i) $\begin{bmatrix}3\\-2\end{bmatrix}$\\
$(3,-2)=a(1,0)+b(0,1) \Longrightarrow a=3,\quad b=-2$\\
b) ii) $\begin{bmatrix}-\frac{5}{5}\\\frac{1}{2}\end{bmatrix}$\\
$(3,-2)=a(-1,1)+b(1,1) \Longrightarrow a=-\frac{5}{2},\quad b=\frac{1}{2}$\\
b) iii) $\begin{bmatrix}\frac{-2+\sqrt{3}}{2}\\\frac{2+\sqrt{3}}{2}\end{bmatrix}$\\
$(3,-2)=a(\sqrt{3},1)+b(\sqrt{3},-1) \Longrightarrow a=\frac{-2+\sqrt{3}}{2},\quad b=\frac{2+\sqrt{3}}{2}$\\
b) iv) $\begin{bmatrix}\frac{3}{2}\\-1\end{bmatrix}$\\
$(3,-2)=a(2,0)+b(0,2) \Longrightarrow a=\frac{3}{2},\quad b=-1$\\
c) i) $\begin{bmatrix}-4\\0\end{bmatrix}$\\
c) ii) $\begin{bmatrix}\frac{-2\sqrt{3}+6}{3}\\\frac{-2\sqrt{3}-6}{3}\end{bmatrix}$\\
c) iii) $\begin{bmatrix}-2\\2\end{bmatrix}$\\

31.\\
a)\\
$\begin{bmatrix}\frac{1}{2}&\frac{\sqrt{3}}{2}\\-\frac{\sqrt{3}}{2}&\frac{1}{2}\end{bmatrix}$\\
b)\\
$\begin{bmatrix}\cos(-\frac{\pi}{3})&\sin(-\frac{\pi}{3})\\-\sin(-\frac{\pi}{3})&\cos(-\frac{\pi}{3})\end{bmatrix}$ = $\begin{bmatrix}\frac{1}{2}&-\frac{\sqrt{3}}{2}\\\frac{\sqrt{3}}{2}&\frac{1}{2}\end{bmatrix}$\\

33.\\
$\begin{bmatrix}1&0\\0&0\end{bmatrix}=\begin{bmatrix}a&b\\0&c\end{bmatrix}
\Longrightarrow a=1, \quad b=c=0$\\
$\begin{bmatrix}1&1\\0&0\end{bmatrix}=\begin{bmatrix}a&b\\0&c\end{bmatrix}
\Longrightarrow a=b=1, \quad c=0$\\
$\begin{bmatrix}1&0\\0&0\end{bmatrix}=\begin{bmatrix}a&b\\0&c\end{bmatrix}
\Longrightarrow a=b=c=1$\\\\
$[I]^{\beta^{'}}_{\beta} = \begin{bmatrix}1&1&1\\0&1&1\\0&0&1\end{bmatrix}$\\

35. Matriz Indentidade $I$.\\

\end{document}