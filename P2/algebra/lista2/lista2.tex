\documentclass[12pt]{article}
\usepackage[utf8]{inputenc}
\usepackage[
a4paper,		% papel A4
left=2cm,		% margem esquerda
right=2cm,		% margem direita
top=2cm,		% margem superior
bottom=2cm		% margem inferior
]{geometry}
\usepackage[T1]{fontenc}	
\usepackage{array,latexsym,amssymb}
\usepackage{amsmath,amsfonts,amssymb,amsthm,mathabx,amstext}
\usepackage{dsfont}	% Conjuntos: $\mathds{N, Z, Q, R, C}$
\usepackage{graphicx}
\begin{document}
\begin{center}
	UNIVERSIDADE FEDERAL DA PARAÍBA\\
	Introdução à Álgebra Linear\\
	Segunda Lista de Exercícios\\
	Paulo Ricardo Seganfredo Campana\\
\end{center}

\noindent Questão 1.\\

\noindent a)\\

Sendo $u=(x_{1},y_{1})$ e $v=(x_{2},y_{2})$\\

$T(u+v)=T(x_{1}+x_{2},y_{1}+y_{2})=(x_{1}+x_{2}+2y_{1}+2y_{2},3x_{1}+3x_{2}-y_{1}-y_{2})$\\

$(x_{1}+2y_{1},3x_{1}-y_{1})+(x_{2}+2y_{2},3x_{2}-y_{2})=T(x_{1},y_{1})+T(x_{2},y_{2})=T(u)+T(v)$\\

$T(\alpha u)=T(\alpha x_{1}, \alpha y_{1})=(\alpha x_{1}+2\alpha y_{1},3\alpha x_{1}-\alpha y_{1})=\alpha(x_{1}+2y_{1},3x_{1}-y_{1})$\\

$\alpha T(x_{1},y_{1})=\alpha T(u)\hfill\blacksquare$\\

\noindent b)\\

Sendo $u=(x_{1},y_{1},z_{1})$, $v=(x_{2},y_{2},z_{2})$\\

$T(u+v)=T(x_{1}+x_{2},y_{1}+y_{2},z_{1},z_{2})=(2x_{1}+2x_{2}-y_{1}-y_{2}+z_{1}+z_{2},y_{1}+y_{2}-4z_{1}-4z_{2})$\\

$(2x_{1}-y_{1}+z_{1},y_{1}-4z_{1})+(2x_{2}-y_{2}+z_{2},y_{2}-4z_{2})=T(x_{1},y_{1},z_{1})+T(x_{2},y_{2},z_{2})=T(u)+T(v)$\\

$T(\alpha u)=T(\alpha x_{1},\alpha y_{1},\alpha z_{1})=(2\alpha x_{1}-\alpha y_{1}+\alpha z_{1},\alpha y_{1}-4\alpha z_{1})=\alpha(2x_{1}-y_{1}+z_{1},y_{1}-4z_{1})$\\

$\alpha T(x_{1},y_{1},z_{1})=\alpha T(u)\hfill\blacksquare$\\

\noindent c)\\

$T(A+A')=(A+A')\cdot B=AB+A'B=T(A)+T(A')$\\

$T(\alpha A)=(\alpha A)\cdot B=\alpha(AB)=\alpha T(A)\hfill\blacksquare$\\

\noindent d)\\

$T(A+B)=tr(A+B)=(a_{11}+b_{11}+a_{22}+b_{22}+...+a_{nn}+b_{nn})$\\

$(a_{11}+a_{22}+...+a_{nn})+(b_{11}+b_{22}+...+b_{nn})=tr(A)+tr(B)=T(A)+T(B)$\\

$T(\alpha A)=tr(\alpha A)=(\alpha a_{11}+\alpha a_{22}+...+\alpha a_{nn}) = \alpha(a_{11}+a_{22}+...+a_{nn})=\alpha tr(A)=\alpha T(A)\hfill\blacksquare$\\

\noindent e)\\

Não é transformação pois $T(0)=T(0+0x+0x^{2})=(1,x,x^{2})$ ou seja, $T(0)\neq0\hfill\blacksquare$\\

\noindent Questão 2.\\

$T(1,1,1)=(2,-1,4)\qquad T(1,1,0)=(3,0,1)\qquad T(1,0,0)=(-1,5,1)$\\

$(x,y,z)=a(1,1,1)+b(1,1,0)+c(1,0,0)=(a+b+c,a+b,a)$\\

$\left\{\begin{array}{@{}l@{}}
	a+b+c=x\\
	a+b=y\\
	a=z
\end{array}\right.$\qquad\qquad
$a=z,\qquad b=y-z,\qquad c=x-y$\\

$T(x,y,z)=zT(1,1,1)+(y-z)T(1,1,0)+(x-y)T(1,0,0)$\\

$z(2,-1,4)+(y-z)(3,0,1)+(x-y)T(-1,5,1)$\\

$(2z,-z,4z)+(3y-3z,0,y-z)+(-x+y,5x-5y,x-y)=(-x+4y-z,5x-5y-z,x+3z)$\\

$T(2,4,-1)=(15,-9,-1)$\\

\noindent Questão 3.\\

$T(2v_{1}-4v_{2}+5v_{3})=T(2v_{1})+T(-4v_{2})+T(5v_{3})=2T(v_{1})-4T(v_{2})+5T(v_{3})$\\

$2(1,-1,2)-4(0,3,2)+5(-3,1,2)=(2,-2,4)+(0,-12,-8)+(-15,5,10)=(-13,-9,6)$\\

\noindent Questão 4.\\

$n^n$ pois a transformação $T$ pode levar qualquer $v_{i}$ em qualquer $v_{j}$ incluindo $i=j$, ou seja, cada vetor $v_{i}$ pode ser mapeado a $n$ diferentes vetores, e existem $n$ vetores $v_{i}$, portanto existem $n^n$ transformações.\\

\noindent Questão 5.\\

\noindent a)\\

$Im(T)=\lbrace(a,b,c);a=(4x+y-2z-3t),b=(2x+y+z-4t),c=(6x-9z+9t)\rbrace$\\

$\lbrace(4x+y-2z-3t,2x+y+z-4t,6x-9z+9t)\forall x,y,z,t \in \mathds{R}\rbrace$\\

$\lbrace x(4,2,6)+y(1,1,0)+z(-2,1,-9)+t(-3,-4,9)\forall x,y,z,t \in \mathds{R}\rbrace$\\

$[(4,2,6),(1,1,0),(-2,1,-9),(-3,-4,9)]$\\

$\begin{bmatrix}
	4 & 2 & 6 \\
	1 & 1 & 0 \\
	-1 & 1 & -9 \\
	-3 & -4 & 9
\end{bmatrix}$\qquad
$\begin{bmatrix}
	4 & 2 & 6 \\
	0 & 2 & -9 \\
	0 & -7 & 36 \\
	0 & -1 & 9
\end{bmatrix}$\qquad
$\begin{bmatrix}
	4 & 2 & 6 \\
	0 & -1 & 9 \\
	0 & 0 & 9 \\
	0 & 0 & -27
\end{bmatrix}$\qquad
$\begin{bmatrix}
	4 & 0 & 24 \\
	0 & 1 & -9 \\
	0 & 0 & 9 \\
	0 & 0 & 0
\end{bmatrix}$\qquad
$\begin{bmatrix}
	1 & 0 & 0 \\
	0 & 1 & 0 \\
	0 & 0 & 1 \\
	0 & 0 & 0
\end{bmatrix}$\\

$Im(T)=[(1,0,0),(0,1,0),(0,0,1)]=\mathds{R}^{3}$, os três vetores estão dentro de $Im(T)$\\

\noindent b)\\

$N(T)\Longrightarrow(4x+y-2z-3t,2x+y+z-4t,6x-9z+9t)=0$\\

$\begin{bmatrix}
	4 & 1 & -2 & -3 & 0 \\
	2 & 1 & 1 & -4 & 0 \\
	6 & 0 & -9 & 9 & 0
\end{bmatrix}$\qquad
$\begin{bmatrix}
	2 & 1 & 1 & -4 & 0 \\
	0 & 1 & 4 & -5 & 0 \\
	0 & -1 & -4 & 9 & 0
\end{bmatrix}$\qquad
$\begin{bmatrix}
	2 & 1 & 1 & -4 & 0 \\
	0 & 1 & 4 & -5 & 0 \\
	0 & 0 & 0 & 9 & 0
\end{bmatrix}\Longrightarrow t=0$\\

$\begin{bmatrix}
	2 & 1 & 1 & 0 \\
	0 & 1 & 4 & 0
\end{bmatrix}$\qquad
$\begin{bmatrix}
	2 & 0 & -3 & 0 \\
	0 & 1 & 4 & 0
\end{bmatrix}$\qquad
$\begin{bmatrix}
	8 & 3 & 0 & 0 \\
	0 & 1 & 4 & 0
\end{bmatrix}$\qquad
$\left\{\begin{array}{@{}l@{}}
	8x+3y=0\\
	y+4z=0
\end{array}\right.\Longrightarrow [(8\alpha,-3\alpha,12\alpha,0)]$\\

$dimN+dimIm=dim\mathds{R}^{4}$\qquad\qquad
$1+3=4$\quad\checkmark

\noindent c)\\

Apenas $u=(3,-8,2,0)$ pois $(12-8-4-0,6-8+2+0,18-18+0)=(0,0,0)$\\

\noindent Questão 6.\\

$T(P_{2})=T(a+bx+cx^{2})=x(a+bx+cx^{2})=ax+bx^{2}+cx^{3}$\\

$N(T)\Longrightarrow ax+bx^{2}+cx^{3}=0$ possui única solução trivial $(a,b,c)=(0,0,0)$\qquad $N(T)=0$\\

$Im(T)\Longrightarrow \lbrace ax+bx^{2}+cx^{3}\rbrace=\lbrace a(x)+b(x^{2})+c(x^{3})\rbrace=[(x,x^{2},x^{3})]$\\

\noindent{\Huge Boldrini - Capítulo 5}\\

\noindent Questão 3.\\

$(x,y,z)=a(1,0,0)+b(0,1,0)+c(0,0,1)=(a,b,c)\qquad\qquad x=a,\quad y=b,\quad z=c$\\

$T(x,y,z)=xT(1,0,0)+yT(0,1,0)+zT(0,0,1)=$\\

$x(2,0)+y(1,1)+z(0,-1)=(2x+y,y-z)=(3,2)$\\

$\left\{\begin{array}{@{}l@{}}2x+y=3\\y-z=2\end{array}\right.\qquad\qquad z=1-2x,\quad y=3-2x\qquad\qquad v=(x, 3-2x, 1-2x)$\\

\noindent Questão 4.\\

\noindent a)\\

$(x,y)=a(1,1)+b(0,-2)=(a,a-2b)\qquad\qquad x=a,\quad y=x-2b,\quad b = \dfrac{x-y}{2}$\\

$T(x,y)=xT(1,1)+(\dfrac{x-y}{2})T(0,-2)=x(3,2,1)+(\dfrac{x-y}{2})(0,1,0)=(3x,\dfrac{5x-y}{2},x)$\\

\noindent b)\\

$T(1,0)=\left(3,\dfrac{5}{2},1\right)\qquad\qquad T(0,1)=\left(0,-\dfrac{1}{2},0\right)$\\

\noindent c)\\

$(x,y,z)=a(3,2,1)+b(0,1,0)+c(0,0,1)=(3a,2a+b,a+c)$\\

$a=\dfrac{x}{3},\quad b=y-\dfrac{2x}{3},\quad c=z-\dfrac{x}{3}$\\

$S(x,y,z)=\dfrac{x}{3}S(3,2,1)+(y-\dfrac{2x}{3})S(0,1,0)+(z-\dfrac{x}{3})S(0,0,1)$\\

$\dfrac{x}{3}(1,1)+(y-\dfrac{2x}{3})(0,-2)+(z-\dfrac{x}{3})(0,0)=(\dfrac{x}{3},-2y+\dfrac{5x}{3})$\\

\noindent d)\\

$P(x,y)=S(T(x,y))=S(3x,\dfrac{5x-y}{2},x)=(x,y)$\\

\noindent Questão 5.\\

$T(x,y)=(y,x)$\\

$\begin{bmatrix}0&1\\1&0\end{bmatrix}$\\

\noindent Questão 10.\\

$R=S(T)\qquad T=S^{-1}(R)$\\

$S^{-1}=\begin{bmatrix}4&2&3\\2&1&1\\-7&-3&-5\end{bmatrix}\qquad\qquad T=\begin{bmatrix}8&0&9\\4&0&4\\-13&2&-15\end{bmatrix}$\\

\noindent Questão 11.\\

\noindent a)\\

$(x,y)=a(1,-1)+b(0,2)=(a,2b-a)\qquad\qquad a=x,\quad b=\dfrac{x+y}{2}$\\

$[T]_{\beta}=[T]_{\beta}^{\alpha}[V]_{\alpha}=\begin{bmatrix}1&0\\1&1\\0&-1\end{bmatrix}\begin{bmatrix}x\\\dfrac{x+y}{2}\end{bmatrix}=\begin{bmatrix}x\\\dfrac{3x+y}{2}\\\dfrac{-x-y}{2}\end{bmatrix}$\\

$T=x(1,0,-1)+(\dfrac{3x+y}{2})(0,1,2)+(\dfrac{-x-y}{2})(1,2,0)=(\dfrac{x-y}{2},\dfrac{x-y}{2},2x-y)$\\

\noindent b)\\

$S(1,-1)=(-2,2,1)=a(1,0,-1)+b(0,1,2)+c(1,2,0)=(a+c,b+2c,2b-a)$\\

$\left\{\begin{array}{@{}l@{}}a+c=-2\\b+2c=2\\2b-a=1\end{array}\right.\qquad\qquad a=\dfrac{-11}{3},\quad b=\dfrac{-4}{3},\quad c=\dfrac{5}{3}$\\

$S(1,-1)=(4,-2,0)=a(1,0,-1)+b(0,1,2)+c(1,2,0)=(a+c,b+2c,2b-a)$\\

$\left\{\begin{array}{@{}l@{}}a+c=4\\b+2c=-2\\2b-a=0\end{array}\right.\qquad\qquad a=\dfrac{20}{3},\quad b=\dfrac{10}{3},\quad c=\dfrac{-8}{3}$\\

$[S]^{\alpha}_{\beta}=\begin{bmatrix}\dfrac{-11}{3}&\dfrac{20}{3}\\\dfrac{-4}{3}&\dfrac{10}{3}\\\dfrac{5}{3}&\dfrac{-8}{3}\end{bmatrix}$\\

\noindent c)\\

$[T]_{\gamma}=[T]_{\gamma}^{\alpha}[T]_{\alpha}=\begin{bmatrix}1&0\\0&0\\0&1\end{bmatrix}\begin{bmatrix}x\\\dfrac{x+y}{2}\end{bmatrix}=\begin{bmatrix}x\\0\\\dfrac{x+y}{2}\end{bmatrix}$\\

$\gamma=\lbrace(1,1,1),(x,y,z),(-1,-1,2)\rbrace$\\

\noindent Questão 14.\\

\noindent a)\\

$T(1,0,0,0)=(1,0)\quad T(0,1,0,0)=(0,1)\quad T(0,0,1,0)=(0,1)\quad T(0,0,0,1)=(1,0)$\\

$(1,0)=1(1,0)+0(0,1)\quad(0,1)=0(1,0)+1(0,1)$\\

$[T]_{\alpha}^{\beta}=\begin{bmatrix}1&0&0&1\\0&1&1&0\end{bmatrix}$\\

\noindent b)\\

$S(x,y)=\begin{bmatrix}2x+y&x-y\\-x&y\end{bmatrix}=\begin{bmatrix}1&0\\0&1\end{bmatrix}$\\

$x=y\quad x=0\quad y=1\qquad\qquad$ impossível.\\

\noindent Questão 19.\\

\noindent a)\quad$z=0\quad x=y\quad [(1,1,0)]$\\

\noindent b)\quad$dimN(T)+dimIm(T)=dim\mathds{R}^{3}\qquad 1+dimIm(T)=3\qquad dimIm(T)=2$\\

\noindent c)\quad não pois $Im(T)\neq\mathds{R}^{3}$\\

\noindent d)\quad Núcleo é uma reta no $\mathds{R}^{3}$ e Imagem é um plano no $\mathds{R}^{3}$\\

\noindent Questão 20.\\

\noindent a)\quad$T(x,y,z)=(y,z)$\\

\noindent b)\quad não existe pois $dim\mathds{R}^{3}>dim\mathds{R}^{2}$\\

\noindent c)\quad$L(x,y,z)=(0,0)$\\

\noindent d)\quad$M(x,y)=(0,0)$\\

\noindent e)\quad$H(x,y,z)=(0,0,0)$\\

\noindent Questão 22.\\

$D(a_{3}x^{3}+a_{2}x^{2}+a_{1}x+a_{0})=(6a_{3}x+2a_{2})$\\

$D(a+b)=(6(a_{3}+b_{3})x+2(a_{2}+b_{2}))=(6a_{3}x+2a_{2})+(6b_{3}x+2b_{2})=D(a)+D(b)$\\

$D(\alpha a)=(6\alpha a_{3}x+2\alpha a_{2})=\alpha(6a_{3}x+2a_{2})=\alpha D(a)$\\

$N(D)=(0+0+a_{1}x+a_{0})=\lbrace x,1\rbrace$\\

\noindent{\Huge Segunda lista}\\

\noindent Questão 1.\\

\noindent a) $T(x,y)=(2x+2x,x+3y)\qquad\qquad\qquad\text{\hspace{+14pt}}T(-2,1)=(-2,1)\qquad T(v)=\lambda v\quad(\lambda=1)\quad\checkmark$\\

\noindent b) $T(x,y,z)=(x+y+z,2y+z,2y+3z)\qquad T(1,1,2)=(4,4,8)\qquad T(v)=\lambda v\quad(\lambda=4)\quad\checkmark$\\

\noindent Questão 2.\\

\noindent a)\quad$[T]_{\alpha}=\begin{bmatrix}1&2\\-1&4\end{bmatrix}$\\

$det\begin{vmatrix}1-\lambda&2\\-1&4-\lambda\end{vmatrix}=0\qquad\qquad\lambda^{2}-5\lambda+6=0\qquad\qquad\lambda=\lbrace2,3\rbrace$\\

$\lambda=2\quad\begin{vmatrix}-1&2\\-1&2\end{vmatrix}=\left\{\begin{array}{@{}l@{}}-x+2y=0\\-x+2y=0\end{array}\right.\qquad\qquad x=2y\qquad\qquad v=[(2,1)]$\\

$\lambda=3\quad\begin{vmatrix}-2&2\\-1&1\end{vmatrix}=\left\{\begin{array}{@{}l@{}}-2x+2y=0\\-x+y=0\end{array}\right.\qquad\qquad x=y\qquad\qquad v=[(1,1)]$\\

\noindent b)\quad$[T]_{\alpha}=\begin{bmatrix}0&1\\-1&0\end{bmatrix}$\\

$det\begin{vmatrix}0-\lambda&1\\-1&0-\lambda\end{vmatrix}=0\qquad\qquad\lambda^{2}+1=0\qquad\qquad\lambda\notin\mathds{R}$\\

\noindent c)\quad$[T]_{\alpha}=\begin{bmatrix}1&1&1\\0&2&1\\0&2&3\end{bmatrix}$\\

$det\begin{vmatrix}1-\lambda&1&1\\0&2-\lambda&1\\0&2&3-\lambda\end{vmatrix}=0\qquad\qquad(1-\lambda)(2-\lambda)(3-\lambda)-2(1-\lambda)$\\

$(1-\lambda)((2-\lambda)(3-\lambda)-2)\qquad\qquad(1-\lambda)(\lambda^{2}-5\lambda+4)\qquad\qquad\lambda=\lbrace1,4\rbrace$\\

$\lambda=1\quad\begin{vmatrix}0&1&1\\0&1&1\\0&2&2\end{vmatrix}=\left\{\begin{array}{@{}l@{}}y+z=0\\y+z=0\\2y+2z=0\end{array}\right.\qquad\qquad z=-y\qquad\qquad v=[(1,0,0),(0,1,-1)]$\\

$\lambda=4\quad\begin{vmatrix}-3&1&1\\0&-2&1\\0&2&-1\end{vmatrix}=\left\{\begin{array}{@{}l@{}}-3x+y+z=0\\-2y+z=0\\2y-z=0\end{array}\right.\qquad\qquad 2y=z, y=x\qquad\qquad v=[(1,1,2)]$\\

\noindent Questão 3.\\

\noindent a)\\

$T(v)=\lambda v\qquad\qquad T(v)=0\qquad\qquad v\in N(T)\qquad\qquad N(T)\neq0$\\

Se $N(T)\neq0$, T não é injetora, portanto não é bijetora e não é invertível.\\

\noindent b)\quad$detA=detA^{t}\longrightarrow det(A-I\lambda)=det(A^{t}-I\lambda)=0$\quad terão as mesmas soluções.\\

\noindent c)\quad O determinante para matrizes triangulares e diagonais é o produto da diagonal principal.\\

$det\begin{vmatrix}a_{11}-\lambda&a_{12}&\dots&a_{1n}\\0&a_{22}-\lambda&\dots&a_{2n}\\\vdots&\vdots&\ddots&\vdots\\0&0&\dots&a_{nn}-\lambda\end{vmatrix}=(a_{11}-\lambda)(a_{22}-\lambda)\dots(a_{nn}-\lambda)=0\qquad\lambda=\lbrace a_{11},a_{22},\dots,a_{nn}\rbrace$\\

\noindent Questão 4.\\

\noindent a)\\

$det\begin{vmatrix}1-\lambda&0&\dots&0\\0&1-\lambda&\dots&0\\\vdots&\vdots&\ddots&\vdots\\0&0&\dots&1-\lambda\end{vmatrix}=(1-\lambda)(1-\lambda)\dots(1-\lambda)=0\qquad\qquad\lambda=1$\\

$\lambda=1\quad\begin{vmatrix}0&0&\dots&0\\0&0&\dots&0\\\vdots&\vdots&\ddots&\vdots\\0&0&\dots&0\end{vmatrix}=\left\{\begin{array}{@{}l@{}}0a+0b+\dots+0z=0\\0a+0b+\dots+0z=0\\\vdots\\0a+0b+\dots+0z=0\end{array}\right.$\\

$v=[(1,0,\dots,0),(0,1,\dots,0),\dots,(0,0,\dots,1)]=\mathds{R}^{n}$\\

\noindent b)\\

$T(\alpha u-\beta v)=\alpha T(u)-\beta T(v)=\alpha(\lambda u)-\beta(\lambda v)$\\

$T(\alpha u-\beta v)=\lambda(\alpha u-\beta v)$\\

\noindent{\Huge Boldrini - Capítulo 6}\\

\noindent Questão 9.\\

$det\begin{vmatrix}1-\lambda&2\\0&-1-\lambda\end{vmatrix}=\lambda^{2}-1=0\qquad\qquad\lambda=\lbrace-1,1\rbrace$\\

$\lambda=-1\quad\begin{vmatrix}2&2\\0&0\end{vmatrix}=\left\{\begin{array}{@{}l@{}}2x+2y=0\\0x+0y=0\end{array}\right.\qquad\qquad y=-x\qquad\qquad v=[(1,-1)]$\\

$\lambda=1\quad\begin{vmatrix}0&2\\0&-2\end{vmatrix}=\left\{\begin{array}{@{}l@{}}2y=0\\-2y=0\end{array}\right.\qquad\qquad y=0\qquad\qquad v=[(1,0)]$\\

\noindent Questão 11.\\

$det\begin{vmatrix}1-\lambda&2&3\\0&1-\lambda&2\\0&0&1-\lambda\end{vmatrix}=(1-\lambda)^{3}=0\qquad\qquad\lambda=1$\\

$\lambda=1\quad\begin{vmatrix}0&2&3\\0&0&2\\0&0&0\end{vmatrix}=\left\{\begin{array}{@{}l@{}}2y+3z=0\\2z=0\\0=0\end{array}\right.\qquad\qquad z=0,y=0\qquad\qquad v=[(1,0,0)]$\\

\noindent Questão 14.\\

$det\begin{vmatrix}1-\lambda&1&2\\1&2-\lambda&1\\2&1&1-\lambda\end{vmatrix}=(1-\lambda)^{2}(2-\lambda)+2+2-(8-4\lambda+1-\lambda+1-\lambda)$\\

$(1-\lambda)^{2}(2-\lambda)-6(1-\lambda)=(1-\lambda)((1-\lambda)(2-\lambda)-6)=(1-\lambda)(\lambda^{2}-3\lambda-4)$\\

$(1-\lambda)(-1-\lambda)(4-\lambda)=0\qquad\qquad\lambda=\lbrace-1,1,4\rbrace$\\

$\lambda=-1\quad\begin{vmatrix}2&1&2\\1&3&1\\2&1&2\end{vmatrix}=\left\{\begin{array}{@{}l@{}}2x+y+2z=0\\x+3y+z=0\\2x+y+2z=0\end{array}\right.\qquad\qquad y=0,z=-x\qquad\qquad v=[(1,0,-1)]$\\

$\lambda=1\quad\begin{vmatrix}0&1&2\\1&1&1\\2&1&0\end{vmatrix}=\left\{\begin{array}{@{}l@{}}y+2z=0\\x+y+z=0\\2x+y=0\end{array}\right.\qquad\qquad y=-2x,z=x \qquad\qquad v=[(1,-2,1)]$\\

$\lambda=4\quad\begin{vmatrix}-3&1&2\\1&-2&1\\2&1&-3\end{vmatrix}=\left\{\begin{array}{@{}l@{}}-3x+y+2z=0\\x-2y+z=0\\2x+y-3z=0\end{array}\right.\qquad\qquad y=z,x=y \qquad\qquad v=[(1,1,1)]$\\

\noindent Questão 18.\\

$det\begin{vmatrix}2-\lambda&0&1&0\\0&2-\lambda&0&1\\12&0&3-\lambda&0\\0&-1&0&0-\lambda\end{vmatrix}=\dots=(\lambda-1)^2(\lambda+1)(\lambda-6)=0\qquad\qquad\lambda=\lbrace-1,1,6\rbrace$\\

$\lambda=-1\quad\begin{vmatrix}3&0&1&0\\0&3&0&1\\12&0&4&0\\0&-1&0&1\end{vmatrix}=\left\{\begin{array}{@{}l@{}}3x+z=0\\3y+t=0\\12x+4z=0\\-y+t=0\end{array}\right.\qquad y=t=0,z=-3x\qquad v=[(1,0,-3,0)]$\\

$\lambda=1\quad\begin{vmatrix}1&0&1&0\\0&1&0&1\\12&0&2&0\\0&-1&0&-1\end{vmatrix}=\left\{\begin{array}{@{}l@{}}x+z=0\\y+t=0\\12x+2z=0\\-y-t=0\end{array}\right.\qquad z=-x=0,t=-y\qquad v=[(0,1,0,-1)]$\\

$\lambda=6\quad\begin{vmatrix}-4&0&1&0\\0&-4&0&1\\12&0&-3&0\\0&-1&0&-6\end{vmatrix}=\left\{\begin{array}{@{}l@{}}-4x+z=0\\-4y+t=0\\12x-3z=0\\-y-6t=0\end{array}\right.\qquad t=4y=0,z=4x\qquad v=[(1,0,4,0)]$\\

\noindent Questão 20.\\

\noindent a)\quad$T(\alpha v)=\alpha T(v)=\alpha(\lambda v)=\lambda(\alpha v)$\\

\noindent b)\\

$v=\lbrace v\in V;T(v)=\lambda v\rbrace\cup\lbrace0\rbrace,x,y\in v$\\

$T(x+\alpha y)=T(x)+\alpha T(y)=\lambda x+\alpha \lambda y=\lambda(x+\alpha y)\qquad\qquad x+\alpha y \in v,\forall v,\forall \alpha \in \mathds{R}$\\

\noindent Questão 25.\\

\noindent a)\quad$T(v)=\lambda v\qquad\qquad T(v)=0\qquad\qquad v\in N(T)\qquad\qquad N(T)\neq0\longrightarrow$ T não é injetora.\\

\noindent b)\quad Sim, se T não é injetora, $\exists v\in V\neq0;v\in N(T)\longrightarrow T(v)=0\longrightarrow0=0v\longrightarrow\lambda=0$\\

\noindent{\Huge Boldrini - Capítulo 7}\\

\noindent Questão 3.\\

\noindent a)\\

$det\begin{vmatrix}2-\lambda&1&0&0\\0&2-\lambda&0&0\\0&0&2-\lambda&0\\0&0&0&3-\lambda\end{vmatrix}=0\qquad\qquad(2-\lambda)^{3}(3-\lambda)=0\qquad\qquad\lambda=\lbrace2,3\rbrace$\\

$\lambda=2\quad\begin{vmatrix}0&1&0&0\\0&0&0&0\\0&0&0&0\\0&0&0&1\end{vmatrix}=\left\{\begin{array}{@{}l@{}}y=0\\t=0\end{array}\right.\qquad\qquad v=[(1,0,0,0),(0,0,1,0)]$\\

$\lambda=3\quad\begin{vmatrix}-1&1&0&0\\0&-1&0&0\\0&0&-1&0\\0&0&0&0\end{vmatrix}=\left\{\begin{array}{@{}l@{}}-x+y=0\\-y=0\\-z=0\end{array}\right.\qquad\qquad v=(0,0,0,1)$\\

$v=[(1,0,0,0),(0,0,1,0),(0,0,0,1)]\quad$ não é base do $\mathds{R}^{4}$ portanto não é diagonalizável.

\noindent b)\\

$m(x)=(2-x)^{2}(3-x)=0$ pois $m(A)=\begin{bmatrix}0&0&0&0\\0&0&0&0\\0&0&0&0\\0&0&0&0\end{bmatrix}$\\

\noindent Questão 5.\\

\noindent a)\\

$(1-\lambda)(a-\lambda)\qquad\qquad\lambda=\lbrace1,a\rbrace\qquad\qquad a\neq1$ pois $\begin{bmatrix}1&1\\0&1\end{bmatrix}$ não é diagonalizável.\\

\noindent b)\\

$(1-\lambda)^{2}\qquad\qquad\begin{bmatrix}0&a\\0&0\end{bmatrix}\qquad\qquad$ só é diagonalizável para $a=0$\\

\noindent Questão 6.\\

\noindent a)\\

$(2-\lambda)(-3-\lambda)^{2}\qquad\qquad \lambda=\lbrace2,-3\rbrace$\\

$\lambda=2\quad\begin{vmatrix}0&0&1\\0&-5&1\\0&0&-5\end{vmatrix}=\left\{\begin{array}{@{}l@{}}z=0\\-5y+z=0\\-5z=0\end{array}\right.\qquad\qquad v=(1,0,0)$\\

$\lambda=-3\quad\begin{vmatrix}-1&0&1\\0&0&1\\0&0&0\end{vmatrix}=\left\{\begin{array}{@{}l@{}}-x+z=0\\z=0\end{array}\right.\qquad\qquad v=(0,1,0)$\\

\noindent b)\\

$[T]_{\beta}=\begin{bmatrix}-3&0&-\frac{5}{2}\\-1&-4&-\frac{7}{2}\\1&1&3\end{bmatrix}\qquad\qquad (2-\lambda)(-3-\lambda)^{2}\quad$ o mesmo polinômio.\\

\noindent c)\quad Não existe, pois $T$ não é diagonalizável já que os autovetores não formam base do $\mathds{R}^{3}$.\\

\noindent Questão 11.\\

\noindent a)\quad $T(v)=T(T(v))\qquad\qquad\lambda v=\lambda^{2}v\qquad\qquad(\lambda^{2}-\lambda)v=0\qquad\qquad\lambda=\lbrace0,1\rbrace$\\

\noindent b)\quad A matriz identidade pois $T(T(x,y))=T(x,y)=(x,y)$ e $\lambda=1$\\

\noindent c)\quad O polinômio característico sera da forma $-\lambda^{n}(1-\lambda)^{m}$ e o polinômio mínimo da forma $-x(1-x)=x(x-1)$ ou seja, fatores lineares distintos, portanto é diagonalizável.\\

\noindent Questão 12.\\

$det\begin{vmatrix}3-\lambda&0&0\\0&2-\lambda&-5\\0&1&-2-\lambda\end{vmatrix}=0\qquad\qquad(3-\lambda)(2-\lambda)(-2-\lambda)+5(3-\lambda)=0$\\

$(3-\lambda)(2-\lambda)(-2-\lambda)+5)=0\qquad\qquad(3-\lambda)(\lambda^{2}+1)\qquad\qquad\lambda=3$\\

$\lambda=3\quad\begin{vmatrix}0&0&0\\0&-1&-5\\0&1&-5\end{vmatrix}\left\{\begin{array}{@{}l@{}}-y-5z=0\\y-5z=0\end{array}\right.\qquad\qquad y=z=0\qquad\qquad v=(1,0,0)$\\

$\lambda=i\quad\begin{vmatrix}3-i&0&0\\0&2-i&-5\\0&1&-2-i\end{vmatrix}\left\{\begin{array}{@{}l@{}}(3-i)x=0\\(2-i)y-5z=0\end{array}\right.\qquad\qquad y=z=0\qquad\qquad v=(1,0,0)$\\

Que por si só não forma base, portanto não é diagonalizável, porém há soluções complexas para $(\lambda^{2}+1),\quad\lambda=\pm i$, o autovetor $(1,0,0)$ junto com os autovetores de $\lambda=\pm i$ formam base portanto $A$ pode ser diagonalizável com uma base complexa.\\ 


\end{document}